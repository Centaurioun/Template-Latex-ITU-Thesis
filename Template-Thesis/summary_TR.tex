\"Ozet haz\i rlan\i rken 1 sat\i r bo\c{s}luk b\i rak\i l\i r. T\"urk\c{c}e
tezlerde, T\"urk\c{c}e \"ozet 300 kelimeden az olmamak kayd\i yla 1-3 sayfa,
{\.I}ngilizce geni\c{s}letilmi\c{s} \"ozet de 3-5 sayfa aras\i nda 
olmal\i d\i r. 

{\.I}ngilizce tezlerde ise, {\.I}ngilizce \"ozet 300 kelimeden az olmamak
kayd\i yla 1-3 sayfa, T\"urk\c{c}e geni\c{s}letilmi\c{s} \"ozet de 
3-5 sayfa aras\i nda olmal\i d\i r. 

\"Ozetlerde tezde ele al\i nan konu k\i saca tan\i t\i larak, kullan\i lan
y\"ontemler ve ula\c{s}\i lan sonu\c{c}lar belirtilir.

\"Ozetlerde kaynak, \c{s}ekil, \c{c}izelge verilmez.

\"Ozetlerin ba\c{s}\i nda, birinci dereceden ba\c{s}l\i k format\i nda
tezin ad{\i} (\"once 72, sonra 18 punto aral\i k b{\i}rak{\i}larak ve 1
sat\i r aral\i kl{\i} olarak) yaz\i lacakt\i r. Ba\c{s}l\i \u{g}\i n
alt\i na b\"uy\"uk harflerle sayfa ortalanarak (T\"urk\c{c}e \"ozet i\c{c}in)
{\bf \"OZET} ve ({\.I}ngilizce \"ozet i\c{c}in) {\bf SUMMARY}
yaz\i lmal\i d\i r.

T\"urk\c{c}e tezlerde T\"urk\c{c}e \"ozetin \.Ingilizce \"ozetten 
\"once olmas{\i} \"onerilir.